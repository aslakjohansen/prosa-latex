\newcommand{\componentA}{0.88}
\newcommand{\componentB}{0.96}
\newcommand{\componentC}{0.997}

\definecolor{macros}{rgb}{\componentA, \componentB, \componentC}
\definecolor{tikz}{rgb}{  \componentB, \componentC, \componentA}
\definecolor{auto}{rgb}{  \componentB, \componentA, \componentC}
\definecolor{next}{rgb}{  \componentA, \componentC, \componentB}
\definecolor{errors}{rgb}{    \componentC, \componentA, \componentB}
\definecolor{templates}{rgb}{ \componentC, \componentB, \componentA}
\definecolor{exercises}{rgb}{ \componentA, \componentB, \componentC}

\newcommand{\bgcolor}{Purple} % this should be renewed below

%%%%%%%%%%%%%%%%%%%%%%%%%%%%%%%%%%%%%%%%%%%%%%%%%%%%%%%%%%%%%%%
%%%%%%%%%%%%%%%%%%%%%%%%%%%%%%%%%%%%%%%%%%%%%%%%%%%%%%%% Macros

% background color pre
{
\setbeamercolor{background canvas}{bg=macros}
\renewcommand{\bgcolor}{macros}

\section{Macros}
\begin{frame}
  \vspace{25mm}
  \begin{center}
    \Huge{Part 1:\\Macros}
  \end{center}
\end{frame}

\subsection{Introduction}
\begin{frame}[fragile]
  \frametitle{Introduction}
  \vspace{3mm}
  \textbf{Q:} What is a macro?
  
  \pause
  \vspace{5mm}
  \textbf{A:} It is a bit like a function, but the evaluation of the body is done by replacing all occurrences of the parameters with the concrete arguments. The resulting text is inserted in the macro calls place.
  
  \pause
  \vspace{5mm}
  \textbf{Consequence:} It can be tricky to think about nested macros, and worse to reason about errors in them.
  
  \pause
  \vspace{5mm}
  As the old adage goes, if your macro definitions represents your peak brilliance, then you are mentally incapable of debugging them.
\end{frame}

\subsection{Keyword Markup}
\begin{frame}[fragile]
  \frametitle{Keyword Markup}
  \vspace{3mm}
  \textbf{Pattern:}
  
  \vspace{5mm}
  \textbf{Definition:}
  
  \vspace{5mm}
  \textbf{Use:}
  
  \vspace{5mm}
  \textbf{Appearance:}
\end{frame}

\subsection{Highlights}
\begin{frame}[fragile]
  \frametitle{Highlights}
  \vspace{3mm}
  \textbf{Pattern:} I need to refer to something of a particular type (e.g., a file, variable or type), and want it to stand out.
  
  \vspace{5mm}
  \textbf{Definition:}
  \begin{minted}[breaklines]{latex}
\newcommand{\filename}[1]{ \textcolor{purple}{\texttt{#1}} }
  \end{minted}
  
  \vspace{5mm}
  \textbf{Use:}
  \begin{minted}[breaklines]{latex}
The configuration is pulled from the \filename{conf.yaml} file.
  \end{minted}
  
  \vspace{5mm}
  \textbf{Appearance:}
  \\
  The configuration is pulled from the \textcolor{purple}{\texttt{conf.yaml}} file.
\end{frame}

\subsection{Description Lists}
\begin{frame}[fragile]
  \frametitle{Description Lists}
  \vspace{3mm}
  \textbf{Pattern:}
  
  \vspace{5mm}
  \textbf{Definition:}
  \begin{minted}[breaklines]{latex}
\newcommand{\textdesc}[1]{ \textit{\textbf{#1}} }
\newcommand{\descitem}[1]{ \item \textdesc{#1} }
  \end{minted}
  
  \vspace{5mm}
  \textbf{Use:}
  
  \vspace{5mm}
  \textbf{Appearance:}
\end{frame}

\subsection{Inspirational Quotes}
\begin{frame}[fragile]
  \frametitle{Inspirational Quotes [1/2]}
  \vspace{3mm}
  \textbf{Pattern:} I want to sprinkle inspirational quotes around my document.
  
  \vspace{5mm}
  \textbf{Definition:}
  \begin{minted}[breaklines]{latex}
\newenvironment{inspiration}[2][0.9]
{
  \begin{center}
  \newcommand{\saveme}{#2}
  \begin{minipagewithmarginpars}{#1\textwidth}
}
{
  
  \raggedleft{--- \textsl{\saveme}}
  \end{minipagewithmarginpars}
  \end{center}
}
  \end{minted}
\end{frame}
\begin{frame}[fragile]
  \frametitle{Inspirational Quotes [2/2]}
  \vspace{3mm}
  \textbf{Use:}
  
  \vspace{5mm}
  \textbf{Appearance:}
\end{frame}

\subsection{Typed Boxes}
\begin{frame}[fragile]
  \frametitle{Typed Boxes [1/2]}
  \vspace{3mm}
  \textbf{Pattern:} I want certain text to be placed in special boxes (e.g., info or conclusion).
  
  \vspace{5mm}
  \textbf{Definition:}
  \begin{minted}[breaklines,fontsize=\footnotesize]{latex}
\newenvironment{tbox}[2][0.9]
{
  \begin{center}
    \begin{tabular}{|p{#1\textwidth}|}
      \hline
      \cellcolor[gray]{0.9}
      \textbf{#2} \\
      \hline
      \cellcolor[gray]{0.95}
}{
      \\
      \hline
    \end{tabular}
  \end{center}
}
  \end{minted}
\end{frame}
\begin{frame}[fragile]
  \frametitle{Typed Boxes [2/2]}
  \vspace{3mm}
  \textbf{Use:}
  
  \vspace{5mm}
  \textbf{Appearance:}
\end{frame}

% background color post
}

%%%%%%%%%%%%%%%%%%%%%%%%%%%%%%%%%%%%%%%%%%%%%%%%%%%%%%%%%%%%%
%%%%%%%%%%%%%%%%%%%%%%%%%%%%%%%%%%%%%%%%%%%%%%%%%%%%%%%% tikz

% background color pre
{
\setbeamercolor{background canvas}{bg=tikz}
\renewcommand{\bgcolor}{tikz}

\section{TikZ}
\begin{frame}
  \vspace{25mm}
  \begin{center}
    \Huge{Part 2:\\\TikZ}
  \end{center}
\end{frame}

\subsection{Coordinate System}
\begin{frame}[fragile]
  \frametitle{Coordinate System}
  \vspace{3mm}
  \begin{center}
    \begin{tikzpicture}[]
      \newcommand{\size}[0]{ 6cm }
      \tikzstyle{dedge} = [thick,->,>=stealth,draw=black]
      \draw[dedge] (0,0) -- (\size,0) node[midway,below]  {\textsl{\textcolor{purple}{x}}};
      \draw[dedge] (0,0) -- (0,\size) node[midway,left] {\textsl{\textcolor{purple}{y}}};
    \end{tikzpicture}
  \end{center}
\end{frame}

\subsection{Nodes}
\begin{frame}[fragile]
  \frametitle{Nodes}
  \vspace{3mm}
  
\end{frame}

\subsubsection{Anchor Points}
\begin{frame}[fragile]
  \frametitle{Nodes \subpart{Anchor Points}}
  \vspace{3mm}
  \begin{tikzpicture}[remember picture, overlay]
    \newcommand{\pointname}[1]{ \texttt{#1} }
    \tikzstyle{dedge} = [thick,->,>=stealth,draw=black]
    
    \tikzstyle{point}=[
      circle,
      draw=purple,
      fill=purple!5,
      anchor=center,
      minimum size=1.2mm,
      inner sep=0pt,
      thick,
    ]
    
    \coordinate (squarecoord) at ([xshift=-3.8cm] current page.center);
    \coordinate (circlecoord) at ([xshift=3.8cm] current page.center);
    
    \node[rectangle,draw,minimum height=2.6cm,minimum width=2.6cm] (square) at (squarecoord) {};
    \node[circle,draw,minimum height=2.6cm,minimum width=2.6cm] (circle) at (circlecoord) {};
    
    \only<2>{
      \node[point] (sc) at (square.center) {};
      \node[point] (cc) at (circle.center) {};
      \node[anchor=north] () at (sc.south) {\pointname{center}};
      \node[anchor=north] () at (cc.south) {\pointname{center}};
      
      \node[point] (sn) at (square.north) {};
      \node[point] (cn) at (circle.north) {};
      \node[anchor=south] () at (sn.north) {\pointname{north}};
      \node[anchor=south] () at (cn.north) {\pointname{north}};
      
      \node[point] (ss) at (square.south) {};
      \node[point] (cs) at (circle.south) {};
      \node[anchor=north] () at (ss.south) {\pointname{south}};
      \node[anchor=north] () at (cs.south) {\pointname{south}};
      
      \node[point] (se) at (square.east) {};
      \node[point] (ce) at (circle.east) {};
      \node[anchor=west] () at (se.east) {\pointname{east}};
      \node[anchor=west] () at (ce.east) {\pointname{east}};
      
      \node[point] (sw) at (square.west) {};
      \node[point] (cw) at (circle.west) {};
      \node[anchor=east] () at (sw.west) {\pointname{west}};
      \node[anchor=east] () at (cw.west) {\pointname{west}};
      
      \node[point] (snw) at (square.north west) {};
      \node[point] (cnw) at (circle.north west) {};
      \node[anchor=south east] () at (snw.north west) {\pointname{north west}};
      \node[anchor=south east] () at (cnw.north west) {\pointname{north west}};
      
      \node[point] (sne) at (square.north east) {};
      \node[point] (cne) at (circle.north east) {};
      \node[anchor=south west] () at (sne.north east) {\pointname{north east}};
      \node[anchor=south west] () at (cne.north east) {\pointname{north east}};
      
      \node[point] (ssw) at (square.south west) {};
      \node[point] (csw) at (circle.south west) {};
      \node[anchor=north east] () at (ssw.south west) {\pointname{south west}};
      \node[anchor=north east] () at (csw.south west) {\pointname{south west}};
      
      \node[point] (sse) at (square.south east) {};
      \node[point] (cse) at (circle.south east) {};
      \node[anchor=north west] () at (sse.south east) {\pointname{south east}};
      \node[anchor=north west] () at (cse.south east) {\pointname{south east}};
    }
  \end{tikzpicture}
\end{frame}

\subsection{Paths}
\begin{frame}[fragile]
  \frametitle{Paths}
  \vspace{3mm}
  
\end{frame}

\subsection{Styles}
\begin{frame}[fragile]
  \frametitle{Styles}
  \vspace{3mm}
  
\end{frame}

\subsection{Anchor Points}
\begin{frame}[fragile]
  \frametitle{Anchor Points}
  \vspace{3mm}
  
\end{frame}

\subsection{Coordinates}
\begin{frame}[fragile]
  \frametitle{Coordinates}
  \vspace{3mm}
  
\end{frame}

% background color post
}

%%%%%%%%%%%%%%%%%%%%%%%%%%%%%%%%%%%%%%%%%%%%%%%%%%%%%%%%%%%%%%%
%%%%%%%%%%%%%%%%%%%%%%%%%%%%%%%%%%%%%%%%%%%%%%%%%%%% Automation

% background color pre
{
\setbeamercolor{background canvas}{bg=auto}
\renewcommand{\bgcolor}{auto}

\section{Automation}
\begin{frame}
  \vspace{25mm}
  \begin{center}
    \Huge{Part 3:\\Automation}
  \end{center}
\end{frame}

\subsection{Table Heatmaps}
\begin{frame}[fragile]
  \frametitle{Table Heatmaps}
  \vspace{3mm}
  A heatmap is a form of visualization whereby the values of the cells of a grid is illustrated by some gradient of colors.
  
  \vspace{5mm}
  In this section we will look at one application:
  
  \begin{center}
    \textsl{We are faced with a choice of data structures. Either we use a list or we use a dictionary (aka map). We will be doing a lot of insertions and want to see how fast each of these operations are, and how well they scale.}
  \end{center}
  
  \vspace{5mm}
  \textbf{Note:} In reality this particular problem is likely better solved by looking up a textbook, but it will serve nicely as an example of a type of problem that we might face.
\end{frame}

\subsubsection{Code}
\begin{frame}[fragile]
  \frametitle{Table Heatmaps \subpart{Code}}
  \vspace{0mm}
  \inputminted[fontsize=\tiny]{python}{../src/table_heatmap/code.py}
\end{frame}

\subsubsection{Benchmarking Results}
\begin{frame}[fragile]
  \frametitle{Table Heatmaps \subpart{Benchmarking Results}}
  \vspace{-1mm}
  \inputminted[fontsize=\footnotesize]{text}{../src/table_heatmap/code.txt}
\end{frame}

\subsubsection{Processing Code}
\begin{frame}[fragile]
  \frametitle{Table Heatmaps \subpart{Processing Code}}
  \vspace{0mm}
  
  \begin{multicols}{2}
    \inputminted[fontsize=\tiny]{python}{../src/table_heatmap/process.py}
  \end{multicols}
\end{frame}

\subsubsection{Presentation Result}
\begin{frame}[fragile]
  \frametitle{Table Heatmaps \subpart{Presentation Result}}
  \vspace{24mm}
  \scalebox{0.9}{
    \input{../src/table_heatmap/code.tex}
  }
\end{frame}

\subsubsection{Takeaways}
\begin{frame}[fragile]
  \frametitle{Table Heatmaps \subpart{Takeaways}}
  \vspace{3mm}
  Whenever I change \filename{code.py} or \filename{process.py}, the build system will redo the steps necessary to make the resulting PDF reflect that change.
  
  \vspace{5mm}
  That means:
  \begin{itemize}
    \item The performance numbers will always be true to the latest version of the code (when executed on the build machine).
    \item The presentation of the data will always be true to the latest version of the processing script.
    \item Observing these qualities comes at zero cost for the programmer.
  \end{itemize}
\end{frame}

\subsection{Exercise Difficulty}
\begin{frame}[fragile]
  \frametitle{Exercise Difficulty}
  \vspace{3mm}
  
\end{frame}

\subsubsection{Solution}
\begin{frame}[fragile]
  \frametitle{Exercise Difficulty \subpart{Solution}}
  \vspace{3mm}
  
\end{frame}

\subsubsection{Result}
\begin{frame}[fragile]
  \frametitle{Exercise Difficulty \subpart{Result}}
  \vspace{3mm}
  
\end{frame}

\subsection{Nametag Generation}
\begin{frame}[fragile]
  \frametitle{Nametag Generation}
  \vspace{3mm}
  You got a spreadsheet with 200+ registrations for a conference, and you need to produce nametags for everyone.
  
  \vspace{5mm}
  Participants have registered for a subset of the planned events which are spread out throughout the course of a (work)week.
  
  \vspace{5mm}
  The information that needs to find their way to each individual nametag is spread out across 60+ columns, and the mapping is far from clean.
  
  \vspace{5mm}
  A reference to the full conference program should be placed on each nametag.
  
  \vspace{5mm}
  You have logos for the conference and sponsors.
\end{frame}

\subsubsection{Solution}
\begin{frame}[fragile]
  \frametitle{Nametag Generation \subpart{Solution}}
  \vspace{3mm}
  
  \vspace{5mm}
  Source: \url{https://github.com/aslakjohansen/icsa25-nametag}
\end{frame}

\subsubsection{Result}
\begin{frame}[fragile]
  \frametitle{Nametag Generation \subpart{Result}}
  \vspace{3mm}
  \begin{tikzpicture}[remember picture,overlay]
    \node[rectangle,anchor=center,draw,fill=white] () at ([yshift=-4mm]current page.center) {\includegraphics[scale=0.25]{figs/nametag.pdf}};
  \end{tikzpicture}
\end{frame}

\subsection{Parameterized Interpreter}
\begin{frame}[fragile]
  \frametitle{Parameterized Interpreter}
  \vspace{3mm}
  
\end{frame}

\subsubsection{Solution}
\begin{frame}[fragile]
  \frametitle{Parameterized Interpreter \subpart{Solution}}
  \vspace{3mm}
  
\end{frame}

\subsubsection{Result}
\begin{frame}[fragile]
  \frametitle{Parameterized Interpreter \subpart{Result}}
  \vspace{3mm}
  
\end{frame}

\subsection{Graph Layout}
\begin{frame}[fragile]
  \frametitle{Graph Layout}
  \vspace{3mm}
  You have to include a visual representation of a graph.
  
  \vspace{5mm}
  It has a number of different properties that can be mapped to color, shape \ldots
  
  \vspace{5mm}
  It has a significant number of nodes and connections. This makes the job of layouting the graph non-trivial. Manual layouting is likely to have horrible results.
  
  \vspace{5mm}
  The \textbf{yEd} program is quite good at layouting and can work with XML files.
\end{frame}

\subsubsection{yEd}
\begin{frame}[fragile]
  \frametitle{Graph Layout \subpart{yEd}}
  \vspace{3mm}
  
\end{frame}

\subsubsection{File Format}
\begin{frame}[fragile]
  \frametitle{Graph Layout \subpart{File Format}}
  \vspace{3mm}
  
\end{frame}

\subsubsection{Processing Code}
\begin{frame}[fragile]
  \frametitle{Graph Layout \subpart{Processing Code}}
  \vspace{3mm}
  
\end{frame}

\subsubsection{Result}
\begin{frame}[fragile]
  \frametitle{Graph Layout \subpart{Result}}
  \vspace{3mm}
  
\end{frame}

\subsection{SVG Narratives}
\begin{frame}[fragile]
  \frametitle{SVG Narratives}
  \vspace{3mm}
  
\end{frame}

\subsubsection{Solution}
\begin{frame}[fragile]
  \frametitle{SVG Narratives \subpart{Solution}}
  \vspace{3mm}
  
\end{frame}

\subsubsection{Result}
\begin{frame}[fragile]
  \frametitle{SVG Narratives \subpart{Result}}
  \vspace{3mm}
  
\end{frame}

% background color post
}

%%%%%%%%%%%%%%%%%%%%%%%%%%%%%%%%%%%%%%%%%%%%%%%%%%%%%%%%%%%%%%%
%%%%%%%%%%%%%%%%%%%%%%%%%%%%%%%%%%%%%%%%%%%%%%%%%%%% Next Steps

% background color pre
{
\setbeamercolor{background canvas}{bg=next}
\renewcommand{\bgcolor}{next}

\section{Next Steps}
\begin{frame}
  \vspace{25mm}
  \begin{center}
    \Huge{Part 4:\\Next Steps}
  \end{center}
\end{frame}

\subsection{KaTeX}
\begin{frame}[fragile]
  \frametitle{KaTeX}
  \vspace{3mm}
  KaTeX allows \LaTeX\ to be used on the web, and sometimes through \textsl{markdown}.
  
  \vspace{5mm}
  More info: \url{https://katex.org}
\end{frame}

% background color post
}

%%%%%%%%%%%%%%%%%%%%%%%%%%%%%%%%%%%%%%%%%%%%%%%%%%%%%%%%%%%%%%%
%%%%%%%%%%%%%%%%%%%%%%%%%%%%%%%%%%%%%%%%%%%%%%%%%%%%% Exercises

% background color pre
{
\setbeamercolor{background canvas}{bg=exercises}
\renewcommand{\bgcolor}{exercises}

\section{Exercises}
\begin{frame}
  \vspace{25mm}
  \begin{center}
    \Huge{Part 4:\\Exercises}
  \end{center}
\end{frame}

\subsection{Exercises}
\begin{frame}[fragile]
  \frametitle{Exercises}
  \vspace{3mm}
  \begin{enumerate}
    \descitem{First Macro:} 
  \end{enumerate}
\end{frame}

% background color post
}

