\newcommand{\componentA}{0.88}
\newcommand{\componentB}{0.96}
\newcommand{\componentC}{0.997}

\definecolor{macros}{rgb}{\componentA, \componentB, \componentC}
\definecolor{tikz}{rgb}{  \componentB, \componentC, \componentA}
\definecolor{auto}{rgb}{  \componentB, \componentA, \componentC}
\definecolor{lists}{rgb}{     \componentA, \componentC, \componentB}
\definecolor{errors}{rgb}{    \componentC, \componentA, \componentB}
\definecolor{templates}{rgb}{ \componentC, \componentB, \componentA}
\definecolor{exercises}{rgb}{ \componentA, \componentB, \componentC}

\newcommand{\bgcolor}{Purple} % this should be renewed below

%%%%%%%%%%%%%%%%%%%%%%%%%%%%%%%%%%%%%%%%%%%%%%%%%%%%%%%%%%%%%%%
%%%%%%%%%%%%%%%%%%%%%%%%%%%%%%%%%%%%%%%%%%%%%%%%%%%%%%%% Macros

% background color pre
{
\setbeamercolor{background canvas}{bg=macros}
\renewcommand{\bgcolor}{macros}

\section{Macros}
\begin{frame}
  \vspace{25mm}
  \begin{center}
    \Huge{Part 1:\\Macros}
  \end{center}
\end{frame}

\subsection{Keyword Markup}
\begin{frame}[fragile]
  \frametitle{Keyword Markup}
  \vspace{3mm}
  
\end{frame}

\subsection{Description Lists}
\begin{frame}[fragile]
  \frametitle{Description Lists}
  \vspace{3mm}
  
\end{frame}

\subsection{Quotes}
\begin{frame}[fragile]
  \frametitle{Quotes}
  \vspace{3mm}
  
\end{frame}

% background color post
}

%%%%%%%%%%%%%%%%%%%%%%%%%%%%%%%%%%%%%%%%%%%%%%%%%%%%%%%%%%%%%
%%%%%%%%%%%%%%%%%%%%%%%%%%%%%%%%%%%%%%%%%%%%%%%%%%%%%%%% tikz

% background color pre
{
\setbeamercolor{background canvas}{bg=tikz}
\renewcommand{\bgcolor}{tikz}

\section{TikZ}
\begin{frame}
  \vspace{25mm}
  \begin{center}
    \Huge{Part 2:\\\TikZ}
  \end{center}
\end{frame}

\subsection{Nodes}
\begin{frame}[fragile]
  \frametitle{Nodes}
  \vspace{3mm}
  
\end{frame}

\subsection{Paths}
\begin{frame}[fragile]
  \frametitle{Paths}
  \vspace{3mm}
  
\end{frame}

\subsection{Styles}
\begin{frame}[fragile]
  \frametitle{Styles}
  \vspace{3mm}
  
\end{frame}

\subsection{Anchor Points}
\begin{frame}[fragile]
  \frametitle{Anchor Points}
  \vspace{3mm}
  
\end{frame}

\subsection{Coordinates}
\begin{frame}[fragile]
  \frametitle{Coordinates}
  \vspace{3mm}
  
\end{frame}

% background color post
}

%%%%%%%%%%%%%%%%%%%%%%%%%%%%%%%%%%%%%%%%%%%%%%%%%%%%%%%%%%%%%%%
%%%%%%%%%%%%%%%%%%%%%%%%%%%%%%%%%%%%%%%%%%%%%%%%%%%% Automation

% background color pre
{
\setbeamercolor{background canvas}{bg=auto}
\renewcommand{\bgcolor}{auto}

\section{Automation}
\begin{frame}
  \vspace{25mm}
  \begin{center}
    \Huge{Part 3:\\Automation}
  \end{center}
\end{frame}

\subsection{Table Heatmaps}
\begin{frame}[fragile]
  \frametitle{Table Heatmaps}
  \vspace{3mm}
  A heatmap is a form of visualization whereby the values of the cells of a grid is illustrated by some gradient of colors.
  
  \vspace{5mm}
  In this section we will look at one application:
  
  \begin{center}
    \textsl{We are faced with a choice of data structures. Either we use a list or we use a dictionary (aka map). We will be doing a lot of insertions and want to see how fast each of these operations are, and how well they scale.}
  \end{center}
  
  \vspace{5mm}
  \textbf{Note:} In reality this particular problem is likely better solved by looking up a textbook, but it will serve nicely as an example of a type of problem that we might face.
\end{frame}

\subsubsection{Code}
\begin{frame}[fragile]
  \frametitle{Table Heatmaps \subpart{Code}}
  \vspace{0mm}
  \inputminted[fontsize=\tiny]{python}{../src/table_heatmap/code.py}
\end{frame}

\subsubsection{Benchmarking Results}
\begin{frame}[fragile]
  \frametitle{Table Heatmaps \subpart{Benchmarking Results}}
  \vspace{-1mm}
  \inputminted[fontsize=\footnotesize]{text}{../src/table_heatmap/code.txt}
\end{frame}

\subsubsection{Processing Code}
\begin{frame}[fragile]
  \frametitle{Table Heatmaps \subpart{Processing Code}}
  \vspace{0mm}
  
  \begin{multicols}{2}
    \inputminted[fontsize=\tiny]{python}{../src/table_heatmap/process.py}
  \end{multicols}
\end{frame}

\subsubsection{Presentation Result}
\begin{frame}[fragile]
  \frametitle{Table Heatmaps \subpart{Presentation Result}}
  \vspace{24mm}
  \scalebox{0.9}{
    \input{../src/table_heatmap/code.tex}
  }
\end{frame}

\subsubsection{Takeaways}
\begin{frame}[fragile]
  \frametitle{Table Heatmaps \subpart{Takeaways}}
  \vspace{3mm}
  Whenever I change \filename{code.py} or \filename{process.py}, the build system will redo the steps necessesary to make the resulting PDF reflect that change.
  
  \vspace{5mm}
  That means:
  \begin{itemize}
    \item The performance numbers will always be true to the latest version of the code (when executed on the build machine).
    \item The presentation of the data will always be true to the latest version of the processing script.
    \item Observing these qualities comes at zero cost for the programmer.
  \end{itemize}
\end{frame}

\subsection{Exercise Difficulty}
\begin{frame}[fragile]
  \frametitle{Exercise Difficulty}
  \vspace{3mm}
  
\end{frame}

% background color post
}

%%%%%%%%%%%%%%%%%%%%%%%%%%%%%%%%%%%%%%%%%%%%%%%%%%%%%%%%%%%%%%%
%%%%%%%%%%%%%%%%%%%%%%%%%%%%%%%%%%%%%%%%%%%%%%%%%%%%% Exercises

% background color pre
{
\setbeamercolor{background canvas}{bg=exercises}
\renewcommand{\bgcolor}{exercises}

\section{Exercises}
\begin{frame}
  \vspace{25mm}
  \begin{center}
    \Huge{Part 4:\\Exercises}
  \end{center}
\end{frame}

\subsection{Exercises}
\begin{frame}[fragile]
  \frametitle{Exercises}
  \vspace{3mm}
  \begin{enumerate}
    \descitem{First Macro:} 
  \end{enumerate}
\end{frame}

% background color post
}

